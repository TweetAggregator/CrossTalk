\section{Motivation}
All public messages posted on Twitter are freely accessible through the Internet, allowing for some interesting statistical tools which analyze the way tweets propagate.

\href{http://www.tweetping.net}{Tweetping} already provides a real-time feed display. All the new tweets are put onto a map, which allows interesting views. It is for example possible to see the sun rising above the US while the number of tweets grows from East to West, but it is impossible to read the messages, as there are way too many of them. One thing is lacking from this tool: a filter. Trends inside the Twitter platform already provide real-time feedback on various subjects, but not specifically displayed on a map. This could have interesting use cases, such as comparing what Swiss-German and Swiss-French people think about a popular vote, or analyzing the way people see Barak Obama in swing states. Such a thing is already partially implemented by Trendsmap.

But this tool could be made even more powerful by adding something currently lacking: population-based statistics. The idea here is to search for various trends and try to relate the populations of the users posting about them. Such analyses could for instance be displayed as a Venn diagram on the side of the map, allowing for some interesting ways to observe correlation and intersection between trends. This tool could for example try to answer such questions as: do Swiss citizens speaking about the vote of February 9th also speak about Erasmus? Is it more the case in Zurich or in Lausanne?
